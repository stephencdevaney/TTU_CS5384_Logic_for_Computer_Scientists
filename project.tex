

\documentclass[12pt, letterpaper]{article}

\setlength{\topmargin}{-1.75cm} \setlength{\textheight}{22.5cm}
\setlength{\oddsidemargin}{0.25cm}
\setlength{\evensidemargin}{0.25cm} \setlength{\textwidth}{17.2cm}

\usepackage{amssymb}
\usepackage{graphicx}

\usepackage{palatino, url, multicol} % for multiple columns

\usepackage{pictex}
%% in the .pictex output of xfig, there is command \colo
%% however the old version of pictex may not define this
%% so we define color here as empty
\def \color#1]#2{}


\newcommand{\courseTitle}
{
 CS5384 Logic for Computer
Scientists }

\newcommand{\teamFormDate} 
{11:59pm Oct 3}
\newcommand{\teamFormHelpDate} 
{11:59pm Sept 30}
\newcommand{\dueDate} 
{11:59pm \dueWeek}
\newcommand{\dueWeek} 
{Nov 7}


\newcommand{\courseDate}
{
    Fall 2010
}
\newcommand{\TA}
{
    Arsu
}

\newcommand{\projecttitle} {
  Satisfiability test of clauses and its application
}

\newcommand{\duedate}{
   11:59pm Friday Nov 12.
}

\begin{document}

\newcommand{\hide}[1]{}
\newcommand{\set}[1]{\{#1\}}
\newcommand{\pg}[1]{{\tt #1}}
\newtheorem{definition}{Definition}


%% \noindent test \hfill test


\title{{\bf Project.} \projecttitle}
\date{} 
\maketitle



\input{hwHeader.tex}

\begin{center}
Submit your program(s) and report to blackboard by  {\bf \dueDate}.  No late submission will be accepted. 
\end{center}
We have studied several methods to check (prove) the
unsatisfiability of propositions. In the last decades, significant
progress has been made to build efficient systems, also called {\em
SAT solvers}, to test the satisfiability of a given proposition in
Conjunctive Normal Form (we know that any proposition can be
translated into CNF). SAT solvers have found wide applications
including hardware/software verification problems, diagnosis and
planning problems.

\section{Objectives and deliverable}

In this project, you are expected to
\begin{itemize}
\item Form a team of exactly three people by {\bf \teamFormDate}. If you cannot find a partner, email our TA Vamsi by {\bf \teamFormHelpDate} so that he can help you. 
\item Download a SAT solver and know how to install it and use it on any Unix
(or Linux) machine of our department.
\item Encode the n-queens (n is number input by users of your program) problem into the DIMACS
CNF format and solve it by the SAT solver you have installed. Since
the CNF format could be very large and complex, you may want to
write a program to generate the CNF coding of the problem.
\item Write a compact while clear report 
% on the important components of the algorithm behind modern SAT solvers {\em and} 
on how you encode the problem into the CNF form. Don't include your source code. You can use pseudo-code for your algorithm to generate CNF for n queens. 
\item Interview. There will be an interview of about 10 minute long. Our TA Vamsi will contact your team for the interview time later. 
% \item {\em Send your source files (including a readme.txt) and the report to \TA\ and cc to me by the due date}.
\end{itemize}

% No submission will be accepted 3 days after the deadline. For late submissions, 10\% will be deducted for each day. 


\section{Background materials: file format and SAT solvers}

The CNF input to a SAT solver is usually stored in a file with
DIMACS CNF format. The input file starts with comments (each line
starts with c). The number of variables and the number of clauses is
defined by the line {\tt p cnf \#variables \#clauses}. Each of the
next lines specifies a clause: a positive literal is denoted by the
corresponding number, and a negative literal is denoted by the
corresponding negative number. The last number in a line should be
zero. Here is an example.

\noindent \pg{c An example SAT formula: \set{\set{A, $\neg$ B},
\set{$\neg$ A, B}}}. \\
\pg{c We have 2 propositional letters (variables) A and B.} \\
\pg{c We use number 1 for A and 2 for B.} \\
\pg{c p CNF \#variables \#clauses} \\
\pg{p CNF 2, 2} \\
\pg{1, -2, 0} \\
\pg{-1, 2, 0}

%\begin{verbatim}
%c An example SAT formula: A ^ ~A c p cnf #var #cl c p cnf 1 2 1 0 -1
%0
%\end{verbatim}

You may use the SAT solver miniSAT  (http://minisat.se/Main.html) or whatever solver you like. From the github page (whose link is on the miniSAT website main page) you can find how a binary miniSAT solver is used. 

\section{Apply logic and a proof procedure to solve interesting
problems}

Here are some hints on encoding the n-queens problem into
propositional logic. Consider 3 queens problem. We need to encode
the following constraints 1) no two queens on the same row, 2) no
two queens on the same column, and 3) no two queens on the same
diagonal.

The objective is to find a formula whose satisfiable assignment
gives us a solution to the n-queens problem. One idea is to
introduce a proposition letter (variable) for each position on the
3X3 board. Let $P_{ij}$ denote the propositional letter for position
$(i,j)$ of the board. If $P_{ij}$ is true, there is a queen on
$(i,j)$; otherwise, there is no queen on $(i,j)$.

To express the observation that there is a queen for first row:
\[P_{11} \vee P_{12} \vee P_{13}.\]
To express the constraint that there is no two queens on the first
row: if any $P_{1i}$ is true, then other $P_{1j}$'s of first row
should be false, i.e.,
\[
\begin{array}{c}
  P_{11} \rightarrow \neg P_{12}, \\
  P_{11} \rightarrow \neg P_{13}, \\
  P_{12} \rightarrow \neg P_{11}, \\
  P_{12} \rightarrow \neg P_{13}, \\
  P_{13} \rightarrow \neg P_{11}, \\
  P_{13} \rightarrow \neg P_{12}. \\
\end{array}
\]
You know how to translate $\rightarrow$ connective to $\vee$
connective.

You may note that to solve the n-queens problem, we are specifying
the problem (using logic) instead of trying to find an algorithm to
solve it. Once we work out the formula, the SAT solver can be
employed to produce a solution to the formula (and thus the original
problem) automatically. This project shows how logic and the proof
procedures can be applied to solve academic (and real life)
problems.

\hide{ 
\section{References}

There are many references on SAT solvers. You may take this paper as
a starting point.
http://www.cs.cornell.edu/gomes/papers/SATSolvers-KR-book-draft-07.pdf
}

\end{document}
